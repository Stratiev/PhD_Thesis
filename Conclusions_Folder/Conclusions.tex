
\chapter*{Conclusions}

    In this thesis we have presented the results of a series of investigations focused around finite density Chern-Simons matter theories. We found a symmetry broken \textcolor{red}{(SB)} phase in the $SU(2)$ theory that persists in the conformal phase of quartic coupling going to zero. The ground state exists both in the Wilson-Fisher fixed point and in the free scalar CS theory. This phase is interesting for several reasons. In the context of Fermi-Bose duality, it is important because up to date confirmations of the duality, have not been made in the finite chemical potential regime. Such checks would require that this condensate be taken into account. Another reason is that it teaches us that giving VEVs to gauge fields does not necessarily break rotational invariance, since that can be mitigated by a colo\textcolor{red}{u}r-flavo\textcolor{red}{u}r locking mechanism. Thirdly, field theoretic techniques in conjunction with this configuration constitute a fundamental mechanism for the existence of roton excitations. This is important since, rotons were first postulated to explain the heat capacity of superfluid helium \textcolor{red}{\cite{PhysRev.60.356}} but no mechanism for their generation was proposed. Finally, this SB phase in the $SU(N)$ theory coincides with the non-commutative Chern-Simons description of the \textcolor{red}{f}ractional \textcolor{red}{q}uantum Hall \textcolor{red}{e}ffect.\\
    \indent We showed that vortex creation can be driven by chemical potential only. In other words, we found vortex solutions that do not require a symmetry breaking potential for their existence. Additionally, we found numerical evidence for BPS vortices, for which an analytic bound cannot be found in the traditional way and derived an approximate corresponding bound. Finally, we found a self-consistent ansatz for non-abelian topological Chern-Simons vortices with \textit{fundamental} matter that exist solely in the finite density regime.\\
    \indent For future work, we wish to extend the computation of the quadratic spectrum of the SU(2) theory to the spectrum of the SU(N) theory. We suspect that the non-commutative Chern-Simons description is the gateway to achieving this and potentially solving the large $N$ $SU(N)$ theory exactly. Additionally, we expect that this non-commutativity is somehow intrinsically linked to the fermionic description of the system. We would like to understand how this condensate manifests itself on the fermionic side of the duality. Is a fermion bilinear responsible for the phase transition in the dual description? If yes, what is the precise mechanism? These are interesting questions, whose answers will help us gain a better understanding of Fermi-Bose duality. Furthermore, we would like to understand the non-abelian equations of motion that we derived at the end of Chapter \textcolor{red}{(}\ref{ch:Chapter_3}\textcolor{red}{)}.\\
    \indent In conclusion, I would like to say that the last few years' contributions of high energy physics to the theory of condensed matter have been very exciting. I think that it is fascinating that ideas, models and theories that have arisen from the study of particle collisions, string theory and quantum gravity are now finding their place on the other extreme of the energy spectrum, in the world of ultracold physics. I am happy to have had the opportunity to contribute my infinitesimal share to this vast endeavour.

%\begin{itemize}
%    \item Vortex creation driven by chemical potential only. No SB potential.
%    \item Symmetry broken phase in the $SU(2)$ theory that persists in the conformal phase of quartic coupling going to zero. The ground state exists both in WF fixed point and in the free scalar CS theory.
%    \item Non-commutative gauge theory arising from a ground state that seems to only appear in the finite chemical potential limit.
%    \item Non-abelian Topological Chern-Simons vortices that also seem to only exist in the finite chemical potential regime.
%    \item The Non-commutative description seems to be a gateway into understanding and solving the large $N$ $SU(N)$ theory exactly. In addition, this non-commutativity gives us a hint into how the fermionic description might arise in this sort of system.
%    \item Roton modes.
%    \item Fermion bilinear condensate and Fermi-Bose duality.
%\end{itemize}
