
% this file is called up by thesis.tex
% content in this file will be fed into the main document
    \graphicspath{{Background_Folder/figures/PNG/}{Background_Folder/figures/PDF/}{Background_Folder/figures/}}

%: ———————-- introduction file header ———————--
\chapter{Vortices in U(2) and SU(2) Chern-Simons scalar theories.}

\label{ch:Chapter_4}



Non-abelian vortices have been studied in \cite{deVega:1986eu, Kumar:1986yz, Blazquez-Salcedo:2013roa, NavarroLerida:2009dm}. These are notably different to their abelian counterparts since the scalars appearing in those models, are in the adjoint representation. This allows for the VEV to remain invariant under the center of the gauge group. If $z\in Z(G)$, where $Z(G)$ is the center of $G$, then by definition $z$ commutes with all elements in the group. Thus, if an adjoint scalar $\Phi$ has a non-zero VEV, then under a gauge transformation generated by z, we have $\Phi \rightarrow z\Phi z^{\dag} =\Phi z z^{\dag} = \Phi$. In the above cases, the gauge group is $SU(N)$ with center $Z(SU(N)) =\mathbb{Z}_N$. This means that the action of the gauge group $SU(N)$ on the adjoint scalars reduces to an $SU(N)/Z(SU(N))= SU(N)/\mathbb{Z}_N$. This gives rise to the possibility of a topological vortex, since $\pi_1(SU(N)/\mathbb{Z}_N)=\mathbb{Z}_N$, even though $\pi_1(SU(N))$ is trivial.

In contrast to these studies, we will be interested in fundamental scalars so this argument does not apply directly to our subject matter. However, due to the appearance of a non-zero chemical potential, we have found out that non-zero VEVs in the gauge sector come to play a role. Since the gauge fields are in the adjoint representation, this allows for the same type of argument for topological stability to reappear in a theory with fundamental matter. In the context of Yang-Mills theory, this type of vortex has been explored by \textit{Gorbar, Jia \& Miransky} \cite{Gorbar:2005pi}.\\  
\indent Similarly to the situation that we encountered in Chapter \ref{ch:Chapter_2}, in the non-abelian case, we also observe a non-trivial ground state. This means that we can expect that there will be a circularly symmetric classical solution that interpolates between this ground state at infinity and the zero fields configuration at the centre. And just as we elaborated in the previous paragraph, we expect to have two different types of topological vortices, depending on whether we are interested in the U(2) or SU(2) system.

%\textcolor{red}{Miransky \textit{et al} non-zero chemical potential papers \cite{Gorbar:2005pi, Gusynin2004}}\\
%\textcolor{red}{Non-abelian vortices can also arise through a mechanism such as \cite{Vachaspati:1991cs}}\\
%\textcolor{red} Write up the holonomy


\section{U(2) theory}

The action for the $U(2)$ theory can be written down as follows.
\begin{align}
    S^{\text{SU(2)}}_{\text{CS}} &= \frac{k}{4 \pi} \int d^3x \; \epsilon^{\mu \nu \rho} \; \mathrm{Tr} \left[A_{\mu} \partial_{\nu}A_{\rho}+ \frac{2}{3} A_{\mu} A_{\nu}A_{\rho} \right], \\
    S^{\text{U(1)}}_{\text{CS}} &= \frac{k'}{4 \pi} \int d^3x \; \epsilon^{\mu \nu \rho} B_{\mu} \partial_{\nu} B_{\rho},\\
    S_{\text{matter}}&= - \int d^3x \left[ \left(D_{\mu} \Phi \right)^{\dag} \left(D^{\mu} \Phi \right) + V\left(\Phi^{\dag}\Phi \right)\right], \\
    S &= S^{\text{SU(2)}}_{\text{CS}} + S^{\text{U(1)}}_{\text{CS}} + S_{\text{matter}}, \\
    D_{\mu} &= \partial_{\mu} + A_{\mu}+B_{\mu}.
\end{align}
This leads to the classical equations of motion for the gauge sector.
\begin{align}
    \frac{k}{2 \pi} F_{\mu \nu} = \epsilon_{\mu \nu \lambda} J^{\lambda}\\
    \frac{k'}{2\pi} f_{\mu \nu} = \epsilon_{\mu \nu \lambda} \frac{1}{2} \mathrm{Tr} \left[\hat{J}^{\lambda} \right],
\end{align}
where
\begin{align}
    F_{\mu \nu}&= \partial_{\mu} A_{\nu} - \partial_{\nu}A_{\mu} + [A_{\mu}, A_{\nu}], \qquad f_{\mu \nu} = \partial_{\mu} B_{\nu} - \partial_{\nu} B_{\mu}, \\
    \hat{J}^{\lambda} &= \sqrt{-g} g^{\lambda \nu} \left( \Phi \left( D_{\nu} \Phi\right)^{\dag} - D_{\nu} \Phi \Phi^{\dag}   -2i \mu_B \delta^0_{\nu} \Phi \Phi^{\dag}  \right), \\
    J^{\lambda} &= \hat{J}^{\lambda} - \frac{1}{2}\mathrm{Tr}\left[\hat{J}^{\lambda} \right]\mathbb{I} \qquad \implies \qquad \mathrm{Tr}\left[J^{\lambda}\right]=0.
\end{align}
More explicitly the equations become
\begin{align}
    \frac{k}{2\pi} \left(\partial_r A_{\theta} - \partial_{\theta} A_r + [A_r, A_{\theta}]\right) &= r \big\lbrace (A_0 +B_0+ i \mu_B ) , \Phi \Phi^{\dag} \big\rbrace - \frac{1}{2} \mathrm{Tr} \left[ \hat{J}^0 \right] \\
    \frac{k}{2\pi} \left(\partial_0 A_r - \partial_r A_0 + [A_0, A_r]\right) &= -\frac{1}{r} \left(\partial_{\theta} \Phi \Phi^{\dag} -\Phi \partial_{\theta}\Phi^{\dag} + \big\lbrace (A_{\theta} +B_{\theta}), \Phi \Phi^{\dag} \big\rbrace \right) - \frac{1}{2} \mathrm{Tr} \left[ \hat{J}^{\theta} \right]\\
    \frac{k}{2 \pi} \left(\partial_{\theta} A_0 - \partial_{0} A_{\theta} + [A_{\theta}, A_0]\right) &=-r \left(\partial_{r} \Phi \Phi^{\dag} \Phi \partial_r \Phi^{\dag} + \big\lbrace (A_r +B_r), \Phi \Phi^{\dag} \big\rbrace \right) - \frac{1}{2} \mathrm{Tr} \left[ \hat{J}^r \right]. \\
    \frac{k'}{2 \pi} \left( \partial_r B_{\theta}- \partial_{\theta} B_r\right) &= \frac{1}{2} \mathrm{Tr} [\hat{J}^0]\\
    \frac{k'}{2 \pi} \left( \partial_0 B_{r}- \partial_{r} B_0\right) &= \frac{1}{2} \mathrm{Tr} [\hat{J}^{\theta}]\\
    \frac{k'}{2 \pi} \left( \partial_{\theta} B_{0}- \partial_{0} B_{\theta}\right) &= \frac{1}{2} \mathrm{Tr} [\hat{J}^r]\\
\end{align}


The scalar equations are 
\begin{align}
    \left(D_{\mu} + i\mu \delta_{\mu}^0 \right) \left[\sqrt{-g} \left(D^{\mu}+i \mu g^{\mu 0}\right)\right] \Phi = - \sqrt{-g} \frac{\delta V}{\delta \Phi^{\dag}}. \label{eq:U2_scalar_EOM}
\end{align}


Inserting the ansatz
\begin{align}
    A_{\mu} &= i \left(a_{\mu}^3 \sigma_3 + a_{\mu}^+ \sigma_{+} +a_{\mu}^{-} \sigma_{-} \right), \label{eq:U2_ansatz_non_abelian_gauge} \\
    B_{\mu} &= i \left( a_{\mu} \mathbb{I} \right), \label{eq:U2_ansatz_abelian_gauge}\\
    \Phi &=  e^{i n \theta} \begin{pmatrix}
        0\\
        \varphi(r)\\
    \end{pmatrix}, \qquad n\in \mathbb{Z}, \qquad \varphi(r) \xrightarrow{r \rightarrow \infty} v, \label{eq:U2_ansatz_scalar}
\end{align}
    where
    \begin{align}
        \sigma_3 = \begin{bmatrix}
            1 & 0 \\
            0 & -1 \\
        \end{bmatrix}, \qquad \mathbb{I}= \begin{bmatrix}
            1 & 0\\
            0 & 1\\
        \end{bmatrix} \\
        \sigma_+ =\begin{bmatrix}
            0 & 1\\
            0 & 0\\
        \end{bmatrix}, \qquad \sigma_- = \begin{bmatrix}
            0 &0\\
            1 &0\\
        \end{bmatrix}
    \end{align}
    \begin{align}
        \Phi \Phi^{\dag} = \varphi(r)^2 \begin{pmatrix} 
            0 & 0\\
            0 & 1\\
        \end{pmatrix} = \varphi(r)^2 \sigma_- \sigma_+.
    \end{align}

    And using the following relations in order to simplify the equations of motion.
    \begin{align}
        [\sigma_+, \sigma_-] = \sigma_3, \qquad [\sigma_3, \sigma_+] = 2 \sigma_+, \qquad [\sigma_3, \sigma_-] = -2 \sigma_-.
    \end{align}

    The gauge equations of motion become

    \begin{align}
        \frac{k}{2 \pi} \left(\partial_{r} a_{\theta}^3 - \partial_{\theta} a_r^{(3)} + i \left(a_r^+ a_{\theta}^- - a_r^- a_{\theta}^{+} \right) \right) &= -r \left( a_0 + \mu - a^{(3)}_0 \right)\varphi^2 \\
        \frac{k}{2 \pi} \left(\partial_r a_{\theta}^{+} - \partial_{\theta} a_r^{+} +2i \left(a_r^{(3)} a_{\theta}^{+} - a_r^{+} a_{\theta}^{(3)} \right) \right) &=r a_0^{+} \varphi^2  \\
        \frac{k'}{2 \pi} \left(\partial_r a_{\theta} - \partial_{\theta} a_r \right) &=r \left(a_0 +\mu - a^{(3)}_0 \right) \varphi^2 + r J_0  \\
        \frac{k}{2 \pi}\left(\partial_{0}a_r^{(3)} - \partial_r a_0^{(3)} + i \left(a_0^+ a_r^- - a_0^- a_r^+  \right)  \right) &= \frac{1}{r} \left(a_{\theta}+n - a_{\theta}^{(3)} \right) \varphi^2  \\
        \frac{k}{2 \pi} \left( \partial_0 a_r^+ - \partial_r a_0^+ + 2i \left( a_0^{(3)} a_r^+ -a_0^+ a_r^{(3)} \right) \right) &= -\frac{1}{r} a_{\theta}^+ \varphi^2  \\
        \frac{k'}{2 \pi} \left( \partial_0 a_r - \partial_r a_0 \right) &= -\frac{1}{r} \left(a_{\theta} +n - a_{\theta}^{(3)} \right) \varphi^2 \\
        \frac{k}{2 \pi} \left(\partial_{\theta} a_0^{(3)} - \partial_0 a_{\theta}^{(3)} + i \left(a_{\theta}^+ a_0^- - a_{\theta}^- a_0^{+} \right) \right) &= r \left(a_r - a_r^{(3)}  \right) \varphi^2 \\
        \frac{k}{2 \pi} \left(\partial_{\theta} a_0^+ - \partial_0 a_{\theta}^+ + 2i \left(a_{\theta}^{(3)} a_0^+ - a_{\theta}^+ a_0^{(3)} \right) \right) &=-r a_r^+ \varphi^2\\
        \frac{k'}{2 \pi} \left(\partial_{\theta}a_0 -\partial_{0} a_{\theta} \right) &= r \left(a_r^{(3)}- a_r \right) \varphi^2,
    \end{align}
    where $J_0$ is determined by requiring that the system is neutral when evaluated at the VEV. This implies that
    \begin{align}
        J_0 = -\frac{\pi v^2}{k} \left(\frac{\mu k}{\pi}- v^2 \right) .
    \end{align}
    For static solutions, we may take
    \begin{align}
        a_{\mu}^{(3)} = f_{\mu}(r), \qquad a_{\mu}^+ = g_{\mu}(r)e^{-i \theta},\qquad a_{\mu}^- = g_{\mu}^*(r) e^{i \theta}, \qquad a_{\mu} = a_{\mu}(r). \label{eq:U2_gauge_ansatz_static}
    \end{align}
    With this ansatz, we have
    \begin{align}
        \frac{k}{2\pi} \left(f_{\theta}'(r) +i \left(g_r g_{\theta}^* - g_r^* g_{\theta} \right) \right) &= -r \left(a_0 +\mu - f_0 \right)\phi^2 \\
        \frac{k}{2 \pi} \left(g_{\theta}'(r) - i g_r +2i \left(f_r g_{\theta} -g_r f_{\theta} \right) \right) &=r g_0 \phi^2 \\
        \frac{k'}{2\pi} a_{\theta}' &= r \left(a_0 + \mu -f_0\right)\phi^2 -r J_0\\
        \frac{k}{2\pi} \left(f_0' + i \left(g_0 g_r^* - g_0^* g_r\right)\right)&= \frac{1}{r} \left(a_{\theta} + n - f_{\theta}\right)\phi^2\\
        \frac{k}{2\pi} \left(g_0' - 2i \left(f_0 g_r - g_0 f_r \right)\right) &= \frac{1}{r} g_{\theta} \phi^2\\
        \frac{k'}{2\pi} a_0' &=\frac{1}{r} \left(a_{\theta} +n - f_{\theta}\right) \phi^2 \\
        \frac{i k}{2\pi} \left(g_{\theta}g_0^* - g_{\theta}^* g_0\right) &= r \left(a_r -f_r\right) \phi^2 \\
        \frac{i k}{2\pi} \left(  g_0 - 2 \left(f_{\theta} g_0 - g_{\theta} f_0 \right)\right) &= r g_r \phi^2\\
        r \left(f_r -a_r\right) \phi^2 &= 0.
    \end{align}
    From these equations we can see that we can further restrict the ansatz so that
    \begin{align}
        f_r &=a_r = 0 \\
        g_0(r) = \tilde{g}_0(r) e^{i \alpha}, \qquad g_{\theta}(r) &= \tilde{g}_{\theta}(r) e^{i \alpha}, \qquad g_r(r) =  \tilde{g}_r(r) e^{i \left(\alpha + \frac{\pi}{2} \right)},
    \end{align}
    where all of $\tilde{g}_{\mu} \in \mathbb{R}$. Note the $\frac{\pi}{2}$ shift of the phase of the $g_r$ component relative to the rest of the functions. This is analogous to how the vacuum components of $A_{\theta}$ and $A_r$ relate to one another. The phase factor $e^{i \alpha}$ has to do with the remaining global colour-flavour locked $U(1)$ symmetry in the system.
    The system of equations then simplifies to
    \begin{align}
        \frac{k}{2 \pi } \left(f_{\theta}' - 2\tilde{g}_r \tilde{g}_{\theta}\right) &= -r \left(a_0 + \mu - f_0\right) \phi^2\\
        \frac{k}{2 \pi} \left(\tilde{g}_{\theta}' + \tilde{g}_r - 2 \tilde{g}_r f_{\theta} \right) & = r\tilde{g}_0 \phi^2 \\
        \frac{k'}{2 \pi} a_{\theta}' &= r \left(a_0 +\mu - f_0 \right)\phi^2 - rJ_0 \\
        \frac{k}{2\pi} \left(f_0'+2 \tilde{g}_0\tilde{g}_r\right) &= \frac{1}{r} \left(a_{\theta}+n - f_{\theta}\right)\phi^2 \\
        \frac{k}{2\pi}\left(\tilde{g}_0' + 2 f_0 \tilde{g}_r \right) &=\frac{1}{r} \tilde{g}_{\theta}\phi^2 \\
        \frac{k'}{2\pi} a_0'&= \frac{1}{r} \left(a_{\theta}+n - f_{\theta}\right) \phi^2 \\
        \frac{k}{2\pi} \left(\tilde{g}_0 - 2\left(f_{\theta} \tilde{g}_0 - \tilde{g}_{\theta} f_0\right)\right) &= r \tilde{g}_r \phi^2.
    \end{align}

Inserting the ansatz \ref{eq:U2_ansatz_non_abelian_gauge}, \ref{eq:U2_ansatz_abelian_gauge}, \ref{eq:U2_ansatz_scalar}, \ref{eq:U2_gauge_ansatz_static} into the scalar equation of motion \ref{eq:U2_scalar_EOM} we arrive at
\begin{align}
    \frac{1}{r} \frac{d}{dr} \left(r \varphi\right)+ &\varphi \left( \left(f_0 -a_0 -\mu\right)^2 + |g_0|^2 - \left(a_r -f_r\right)^2 - |g_r|^2 \right)\nonumber \\
    &+ \varphi \left(-\frac{1}{r^2} \left(f_{\theta} - a_{\theta} -n\right)^2 +|g_{\theta}|^2 \right) = - \frac{\delta V}{\delta \Phi^{\dag}}.
\end{align}

We would like a solution that asymptotes to ground state that we explored in Chapter \ref{ch:Chapter_3} (\eqref{E14},\eqref{VEV}). This implies the following boundary conditions
\begin{align}
    \varphi &\xrightarrow{r \rightarrow \infty} v, \qquad \qquad \qquad & \varphi \xrightarrow{r \rightarrow 0} 0,\\
    \tilde{g}_{\theta} &\xrightarrow{r \rightarrow \infty} r \frac{\pi v}{k} \sqrt{ \frac{\mu k}{\pi} -v^2}, &f_{\theta} \xrightarrow{r \rightarrow \infty}0,\\ 
    \tilde{g}_0 &\xrightarrow{r \rightarrow \infty} 0, &f_0  \xrightarrow{r \rightarrow \infty}\frac{\pi v^2}{k},\\
     a_0 &\xrightarrow{r \rightarrow \infty} 0, &a_{\theta} \xrightarrow{r \rightarrow \infty} 0,\\
    & \qquad\qquad\qquad\qquad  \tilde{g}_r \xrightarrow{r \rightarrow \infty} \frac{\pi v}{k} \sqrt{ \frac{\mu k}{\pi} -v^2}.  & 
\end{align}




\section{SU(2) theory}

The $SU(2)$ equations of motion are determined by requiring that the trace of the $U(2)$ equations of motion vanishes. This leads to the simplified system
    \begin{align}
        \frac{k}{2 \pi} \left(\partial_{r} a_{\theta}^3 - \partial_{\theta} a_r^{(3)} + i \left(a_r^+ a_{\theta}^- - a_r^- a_{\theta}^{+} \right) \right) &= -r \left(\mu - a^{(3)} \right)\phi^2 \\
        \frac{k}{2 \pi} \left(\partial_r a_{\theta}^{+} - \partial_{\theta} a_r^{+} +2i \left(a_r^{(3)} a_{\theta}^{+} - a_r^{+} a_{\theta}^{(3)} \right) \right) &=r a_0^{+} \phi^2  \\
        \frac{k}{2 \pi}\left(\partial_{0}a_r^{(3)} - \partial_r a_0^{(3)} + i \left(a_0^+ a_r^- - a_0^- a_r^+  \right)  \right) &= \frac{1}{r} \left(n - a_{\theta}^{(3)} \right) \phi^2  \\
        \frac{k}{2 \pi} \left( \partial_0 a_r^+ - \partial_r a_0^+ + 2i \left( a_0^{(3)} a_r^+ -a_0^+ a_r^{(3)} \right) \right) &= -\frac{1}{r} a_{\theta}^+ \phi^2  \\
        \frac{k}{2 \pi} \left(\partial_{\theta} a_0^{(3)} - \partial_0 a_{\theta}^{(3)} + i \left(a_{\theta}^+ a_0^- - a_{\theta}^- a_0^{+} \right) \right) &= -r  a_r^{(3)} \phi^2 \\
        \frac{k}{2 \pi} \left(\partial_{\theta} a_0^+ - \partial_0 a_{\theta}^+ + 2i \left(a_{\theta}^{(3)} a_0^+ - a_{\theta}^+ a_0^{(3)} \right) \right) &=-r a_r^+ \phi^2.
    \end{align}




    \begin{align}
        \frac{k}{2 \pi } \left(f_{\theta}' - 2\tilde{g}_r \tilde{g}_{\theta}\right) &= -r \left(\mu - f_0\right) \phi^2\\
        \frac{k}{2 \pi} \left(\tilde{g}_{\theta}' + \tilde{g}_r - 2 \tilde{g}_r f_{\theta} \right) & = r\tilde{g}_0 \phi^2 \\
        \frac{k}{2\pi} \left(f_0'+2 \tilde{g}_0\tilde{g}_r\right) &= \frac{1}{r} \left(n - f_{\theta}\right)\phi^2 \\
        \frac{k}{2\pi}\left(\tilde{g}_0' + 2 f_0 \tilde{g}_r \right) &=\frac{1}{r} \tilde{g}_{\theta}\phi^2 \\
        \frac{k}{2\pi} \left(\tilde{g}_0 - 2\left(f_{\theta} \tilde{g}_0 - \tilde{g}_{\theta} f_0\right)\right) &= r \tilde{g}_r \phi^2.
    \end{align}


    \section{Discussion}
    We took the vacuum solutions in the SU(2) and U(2) theories found in Chapter \ref{ch:Chapter_3} and postulated the existence of vortex solutions asymptoting to these vacua. We found a consistent circularly symmetric ansatz and derived the equations of motion. Just as the ground state has an unbroken $U(1)$ symmetry, the solutions of the equations seem to also be parametrized by one real, compact parameter. Since the $SU(N)$ solutions correspond to a winding in the global symmetry, we expect that their energies will diverge in the infinite volume limit. However, in the $U(2)$ case the abelian symmetry is gauged so we expect the vortices in this model to be of finite energy. Further, due to the non-trivial vacuum structure, we expect these solutions to be topologically stable. We leave the numerical analysis of these solutions for further work.



