
% this file is called up by thesis.tex
% content in this file will be fed into the main document
    \graphicspath{{Introduction_Folder/figures/PNG/}{Introduction_Folder/figures/PDF/}{Introduction_Folder/figures/}}

%: ———————-- introduction file header ———————--
\chapter*{Introduction}

On the second day of my graduate studies, The Nobel prize in Physics was awarded to D. Thouless, M. Kosterlitz and D. Haldane for the discovery of topological phase transitions. At the time I thought topology was a subject about open neighbourhoods, Hausdorf spaces and counting holes on mugs and doughnuts. I could not fathom how these ideas could have anything to do with the description of the real world. I had to understand how these concepts were connected. This thesis follows my path of exploring the mystery of topological physics from the perspective of high energy physics and quantum field theory (QFT).

Today, topology has become ubiquitous in the study of physics. It provides us with new models to gain insight into nature. In addition, it provides the stability for new kinds of excitations in models whose initial purpose had more to do with the study of particles (\textit{e.g.} Glashow-Salam model). Even very deep questions such as the understanding of confinement and the attempts to study strongly correlated gauge theories seem to be intimately related to the study of continuous deformations. There are hints that confinement might be related to instanton condensation \cite{Polyakov1977}. And most relevant to this work, it seems that the promising path of relating strongly and weakly coupled theories through dualities, relate the conserved current of a theory to the topological current of its dual.

This will be the introduction chapter. I will talk about:



\begin{itemize}
    \item The reunion of QFT and CMT that has been sparked in recent years by studies such as dualities
    \item Importance of topological field theory in practice. Quantum Computing and \textit{etc}. 
    \item Most importantly, highlight the importance of studying dualities as a tool to understand strongly coupled field theories. Deforming well known dualities and attempting to match the two sides takes us closer to having dual descriptions of theories that are of physical interest.
    \item 
\end{itemize}
