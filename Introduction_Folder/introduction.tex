
% this file is called up by thesis.tex
% content in this file will be fed into the main document
    \graphicspath{{Introduction_Folder/figures/PNG/}{Introduction_Folder/figures/PDF/}{Introduction_Folder/figures/}}

%: ———————-- introduction file header ———————--
\chapter*{Introduction}

%\thechapter
%\pagestyle{fancy}
%\fancyhf{}
%\rhead{Overleaf}
%\lhead{Guides and tutorials}

On the second day of my graduate studies, The Nobel prize in Physics was awarded to D. Thouless, M. Kosterlitz and D. Haldane for the discovery of topological phase transitions\footnote{https://www.nobelprize.org/prizes/physics/2016/summary/}. At the time I thought topology was a subject about open neighbourhoods, hausdorff spaces and counting holes on mugs and doughnuts. I could not fathom how these ideas could have anything to do with the description of the real world. I had to understand how these concepts were connected. This thesis follows my path of exploring the mystery of topological physics from the perspective of high energy physics and quantum field theory (QFT).

Today, topology has become ubiquitous in the study of physics. It provides us with new models to gain insight into nature. In addition, it provides the stability for new kinds of excitations in models whose initial purpose had more to do with the study of particles \textcolor{red}{(For example, topological degrees of freedom were discovered in the Georgi-Glashow model \cite{Georgi:1974sy} by 't Hooft \& Polyakov in 1974 \cite{Polyakov:1974ek, tHooft:1974kcl})}. Even very deep questions such as the understanding of confinement and the attempts to study strongly correlated gauge theories seem to be intimately related to the study of continuous deformations. There are hints that confinement might be related to instanton condensation \cite{Polyakov:1976fu}. And most relevant to this work, it seems that the promising path of relating strongly and weakly coupled theories through dualities, relate the conserved current of a theory to the topological current of its dual. It was our hope that we can shed some more light on the details of these dualities at finite matter density and for this reason, this thesis has been centered around the study of the grand canonical ensemble of Chern-Simons matter theories, specifically scalar field theories.

\textcolor{red}{Before we proceed to list the results of this thesis, we take the time to track down the history of the subject of dualities. In 1975, Sydney Coleman \cite{Coleman:1974bu} showed an equivalence between the \textit{sine-Gordon} model and the massive \textit{Thirring} model in two space-time dimensions, establishing the first known \textit{Fermi-Bose duality}. Later on, this was generalized by Witten \cite{Witten:1983ar} to non-abelian theories in two dimensions. Earlier, in 1941, Kramers and Wannier \cite{Kramers:1941kn} found that the Ising model was dual to itself, under a transformation of the coupling, which allowed them to compute the critical point of the theory, several years before it was solved exactly by Onsager \cite{Onsager:1943jn}. Another example of a duality is also apparent in Maxwell's equations in free space, which possess the property of being invariant under an exchange of electric and magnetic fields ($\bm{E}\rightarrow \bm{B}$, $\bm{B}\rightarrow -\bm{E}$). This duality can be preserved in the presence of sources if we were to also allow for sources of magnetic charge. Dirac took the idea of a magnetic source (\textit{magnetic monopole}) seriously and showed that a singular configuration with this property whose magnetic charge was quantized in reciprocal units of the electric charge \cite{Dirac:1931kp} $eq = 2\pi \hbar n$, where $e$ is the electric charge, $q$ is the magnetic charge of the hypothetical monopole and $n \in \mathbb{Z}$. This idea that duality transforms elementary particles into non-perturbative objects (Here the point electron turns into a dirac string) often plays a prominent role in the dual description of a theory. This is precisely what happens in the non-abelian generalization of electromagnetic duality.}


\textcolor{red}{In 1977, David Olive and Claus Montonen \cite{Montonen:1977sn} proposed that in the Georgi-Glashow model this duality is preserved in the full quantum theory and that the elementary excitations are exchanged with the monopoles discovered by 't Hooft and Polyakov in \cite{Polyakov:1974ek, tHooft:1974kcl}, and vice-versa. Today, this equivalence is known as \textit{S-duality}, also referred to as \textit{strong-weak duality}. Many checks for this conjecture have been made for the supersymmetric version of this theory \cite{Osborn:1979tq, Witten:1978mh, Sen:1994yi, Vafa:1994tf}. Shortly after, Seiberg \cite{Seiberg:1994bz} and Seiberg and Witten \cite{Seiberg:1994rs} solved the $\mathcal{N}=1$ and $\mathcal{N}=2$ Super Yang-Mills (SYM) theories exactly, putting the duality on solid ground for supersymmetric theories in 4 dimensions. This version of S-duality became known as \textit{Seiberg duality}.}

\textcolor{red}{It is a natural question to ask what happens to this duality in a different number of dimensions, particularly of interest to us would be the case of two spatial and one time dimensions. The first such duality in 3d dimensions was found by Aharony in \cite{Aharony:1997gp}. This was later generalized to include a Chern-Simons term by Giveon and Kutasov \cite{Giveon:2008zn}. \cite{Benini:2011m}}

The first novelty that we stumbled upon was a ground state that had been overlooked since it only manifests itself in the finite chemical potential regime. It is a ground state with non-zero expectation value for the gauge fields, seemingly breaking rotational symmetry. In reality, color-flavor locking remedies this and the rotational symmetry is preserved. In order to confirm this, we also computed the quadratic spectrum of the theory, which showed no signs of rotational asymmetry. We discovered further, owing to Goldstone's theorem and the broken $U(1)_B$ global symmetry, that the long wave-length dynamics of the SU(2) theory behaved as a superfluid and predicts the existence of a roton excitation. What was even more surprising is that this non-trivial ground state exists in the zero coupling free scalar theory coupled to Chern-Simons. This is peculiar since we generally expect there to be a potential that drives the symmetry breaking.

Correlation functions for the scalar Chern-Simons theory at $\mu=0$ have been computed exactly in the large $N$ limit \cite{Aharony:2012nh}. Since the $\mu =0$ results cannot be extrapolated to finite $\mu$, we would need to take the ground state discussed in the previous paragraph into account. A natural question that arises is what do the correlation functions look like in the presence of this new condensate at large $N$? This was the ultimate aim that we \textcolor{red}{had} in commencing this work, \textit{i.e.} solving this theory in the large $N$ limit. In order to pursue this direction, we set ourselves the more conservative goal of computing the quadratic spectrum in the general $N$ SU(N) theory. The ground state in the SU(N) theory has the same algebraic structure as the non-commutative Chern-Simons theory used as a model for the Fractional Quantum Hall Effect (FQHE). The approach to diagonalizing the large $N$ spectrum of fluctuations that we outlined involves translating the problem to a non-commutative Chern-Simons theory and using the solutions of that model to arrive at an exact result.

    Aside from studying the ground state and the fundamental excitations of finite density Chern-Simons models, we have also explored the existence and properties of solitons (vortices) in these models. Our main results in this line are the discovery and study of an approximate numerical BPS state in the abelian theory. Additionally, we find a self consistent vortex ansatz for the SU(2) and U(2) theories. We have left the numerical study of the resulting equations for future work.

    This thesis is structured in the following way:

    In Chapter 1, we cover the necessary background required to understand the rest of the thesis and how it fits in the wider field of research. We present a condensed introduction to the topics of Chern-Simons field theory, the Quantum Hall effect, vortices, particle-vortex duality, Fermi-Bose duality, the statistical grand canonical ensemble in quantum field theory and non-commutative field theory.

    Chapter 2 presents the analytical and numerical study of abelian vortex solutions interpolating between a symmetric and asymmetric phase, where the symmetry breaking is chemical potential driven.

    Chapter 3 showcases the derivation of the peculiar ground state mentioned earlier, the spectrum of fluctuations is analyzed in the SU(2) theory. The chapter concludes with the SU(N) ground state and the observation that it resembles non-commutative Chern-Simons theory.

    In Chapter 4 we switch gears back to vortices, except this time we consider the possibility of topological solutions in the non-abelian theory. We show that there is a self-consistent circularly symmetric ansatz for a global vortex in the SU(2) theory and a local vortex in the U(2) theory.


---------------------------------------------------------------------------------------------

A bunch of references that I will need (Particle-Vortex Duality) \cite{Peskin:1977kp, Dasgupta:1981zz, PhysRevB.39.2756} 
(The first example of supersymmetric $\mathcal{N}=2$ Particle-Vortex like duality) \cite{Aharony:1997bx}
$\mathcal{N}=2$ 3d CS matter gauge theories and flow from Giveon Kutasov to Aharony duality.\cite{Intriligator:2013lca}
Aharony Duality \cite{Aharony:1997gp}
Seiberg Duality in 3dimensions \cite{Karch:1997ux}
Mirror Symmetry in $\mathcal{N}=4$ SUSY \cite{Intriligator:1996ex}
General supersymmetric 3d Chern-Simons duality \cite{Aharony:2014uya}
Monopole operators in SUSY CS \cite{Aharony:2015pla}
Non-SUSY dualities have been motivated primarily by the vector model/high spin $AdS_4$ correspondence. \cite{Sezgin:2003pt, Klebanov:2002ja}, Original high spin paper \cite{Vasiliev:1992av}
%This will be the introduction chapter. I will talk about:
%
%
%
%\begin{itemize}
%    \item The reunion of QFT and CMT that has been sparked in recent years by studies such as dualities
%    \item Importance of topological field theory in practice. Quantum Computing and \textit{etc}. 
%    \item Most importantly, highlight the importance of studying dualities as a tool to understand strongly coupled field theories. Deforming well known dualities and attempting to match the two sides takes us closer to having dual descriptions of theories that are of physical interest.
%    \item 
%\end{itemize}
