
% this file is called up by thesis.tex
% content in this file will be fed into the main document
    \graphicspath{{Introduction_Folder/figures/PNG/}{Introduction_Folder/figures/PDF/}{Introduction_Folder/figures/}}

%: ———————-- introduction file header ———————--
\chapter*{Introduction}

%\thechapter
%\pagestyle{fancy}
%\fancyhf{}
%\rhead{Overleaf}
%\lhead{Guides and tutorials}

On the second day of my graduate studies, \textcolor{red}{t}he Nobel prize in \textcolor{red}{p}hysics was awarded to \textit{D. Thouless, M. Kosterlitz \& D. Haldane} for the discovery of topological phase transitions\footnote{https://www.nobelprize.org/prizes/physics/2016/summary/}. At the time I thought topology was a subject about open neighbourhoods, Hausdorff spaces and counting holes on mugs and doughnuts. I could not fathom how these ideas could have anything to do with the description of the real world. I had to understand how these concepts were connected. This thesis follows my path of exploring the mystery of topological physics from the perspective of high energy physics and quantum field theory (QFT).

Today, topology has become ubiquitous in the study of physics. It provides us with new models that give us insight into nature. In addition, topology provides the stability for new kinds of excitations in models whose initial purpose had more to do with the study of particles (for example, topological degrees of freedom were discovered in the \textit{Georgi-Glashow} model \cite{Georgi:1974sy} by \textit{'t Hooft} \textcolor{red}{and} \textit{Polyakov} in 1974 \cite{Polyakov:1974ek, tHooft:1974kcl}). Even very deep ideas such as the understanding of confinement and the attempts to study strongly correlated gauge theories seem to be intimately related to the study of continuous deformations. There are hints that confinement might be related to instanton condensation \cite{Polyakov:1976fu}. \textcolor{red}{Specifically relevant to this work, we find that relating strongly and weakly coupled theories through dualities maps the conserved current of a theory to the topological current of its dual.} It was our hope that we \textcolor{red}{could} shed some more light on the details of these dualities at finite matter density and\textcolor{red}{,} for this reason, this thesis has been centered around the study of the grand canonical ensemble of Chern-Simons matter theories, specifically scalar field theories.

Before we proceed to list the results of this thesis, we take the time to track down the history of the subject of dualities. In 1941, \textit{Kramers and Wannier} \cite{PhysRev.60.252} found that the two dimensional Ising model \textcolor{red}{is} dual to itself, under a transformation of the coupling, which allowed them to compute the critical point of the theory, several years before it was solved exactly by \textit{Onsager} \cite{PhysRev.65.117}.  Then, in 1975, \colorbox{red}{ } \textit{Coleman} \cite{Coleman:1974bu} showed an equivalence between the \textit{sine-Gordon} model \cite{Scott1973} and the massive \textit{Thirring} model \cite{Thirring1958} in two space-time dimensions, establishing the first known \textit{Fermi-Bose duality}. Later on, this was generali\textcolor{red}{s}ed by \textit{Witten} \cite{Witten:1983ar} to non-abelian theories in 1+1 dimensions. Also concerning a 1+1 dimensional system, \textit{level-rank} duality relates two different Wess-Zumino-Witten models with different gauge groups and levels of their \textit{Kac-Moody algebras} \cite{Nakanishi:1990hj, Naculich:1990hg}. Adding an extra dimension, \textit{Naculich, Schnitzer, Riggs \& Mlawer} \cite{Mlawer:1990uv, Naculich:2007nc} showed that the same type of duality applies to pure Chern-Simons theory, a topological \textcolor{red}{quantum} field theory (TQFT) in 2+1 dimensions. 

Another example of a duality dates further back to Maxwell's equations in free space, which possess the property of being invariant under an exchange of electric and magnetic fields
\begin{align}
\bm{E}\rightarrow \bm{B}, \qquad\qquad\qquad \bm{B}\rightarrow -\bm{E},
\end{align}
where $\bm{E}$ and $\bm{B}$ are the electric and magnetic fields, respectively. This duality can be preserved in the presence of sources if we were to also allow for sources of magnetic charge. \textit{Dirac} \cite{Dirac:1931kp} took the idea of a magnetic source (\textit{magnetic monopole}) seriously \textcolor{red}{and showed that if one allows for a singularity (we shall see later that a singularity is actually not necessary) in the vector potential, a configuration with the property of a magnetic monopole can exist. Further still, he proved that electric charge would be quanti\textcolor{red}{s}ed if such a configuration exists}
\begin{align}
eq = 2\pi \hbar n,
\end{align}
where $e$ is the electric charge, $q$ is the magnetic charge of the hypothetical monopole and $n \in \mathbb{Z}$. The idea that duality transforms elementary particles into non-perturbative objects (here the point electron turns into a \textit{Dirac string}) often plays a prominent role in the dual description of a theory. This is precisely what happens in the non-abelian generali\textcolor{red}{s}ation of electromagnetic duality.


In 1977, \textit{Olive \& Montonen} \cite{Montonen:1977sn} proposed that, in the Georgi-Glashow model, electromagnetic duality is preserved in the full quantum theory and that the elementary excitations are exchanged with the monopoles discovered by \textit{'t Hooft} and \textit{Polyakov} \cite{Polyakov:1974ek, tHooft:1974kcl}, and vice-versa. Today, this equivalence is known as \textit{S-duality}, also referred to as \textit{strong-weak duality}. Many checks for this conjecture have been made for the supersymmetric \textcolor{red}{(SUSY)} version of this theory \cite{Osborn:1979tq, Witten:1978mh, Sen:1994yi, Vafa:1994tf}. In 1994, \textit{Seiberg} \cite{Seiberg:1994bz, Seiberg:1994pq} and \textit{Seiberg \& Witten} \cite{Seiberg:1994rs} solved the $\mathcal{N}=1$ and $\mathcal{N}=2$ Super Yang-Mills (SYM) theories exactly, putting the duality on solid ground for supersymmetric theories in 3+1 dimensions. This version of S-duality became known as \textit{Seiberg duality}.

It is a natural question to ask what happens to Seiberg duality in a different number of dimensions, particularly of interest to us, in this work, would be the case of two spatial \textcolor{red}{dimensions} and one time dimension. The first cases of dualities that resemble Seiberg duality in \textcolor{red}{2+1 dimensions} were found by \textit{Aharony} and \textit{Karch} \cite{Aharony:1997gp, Karch:1997ux}. These were later generali\textcolor{red}{s}ed to include a Chern-Simons term by \textit{Giveon \& Kutasov} \cite{Giveon:2008zn}, generali\textcolor{red}{s}ing the level-rank dualities, that we mentioned earlier, to include supersymmetric matter.

Dualities also played a major role in the \textit{second superstring revolution}, when several of the distinct self-consistent string theories were shown to be dual to each other \cite{Sen:1994fa, Schwarz:1994xn, Sathiapalan:1986zb, Hull:1994ys} and possibly dual to an 11-dimensional theory, known as \textit{M-theory} \cite{Witten:1995ex}.

Another example of duality that comes from string theory is one inspired by the study of black holes and the \textit{holographic principle} \cite{Susskind:1994vu, tHooft:1993dmi}. This is the well-studied $AdS/CFT$ correspondence \cite{Aharony:1999ti, Maldacena:1997re}, which states that one can identify operators on the Anti-de Sitter ($AdS$) space with operators on the boundary of the $AdS$ space\textcolor{red}{, which is a conformal field theory (CFT)}. A special duality of this type involves a peculiar 3+1 dimensional gravitational theory on an $AdS_4$ background with a matter content comprising of an infinite tower of higher \textcolor{red}{($>2$)} spin  fields. This higher spin theory is known as \textit{Vasiliev's gravity}, named after its inventor \cite{Vasiliev:1992av}. It turns out that this theory is dual to free $O(N)$ and $SU(N)$ scalar and fermion theories in 2+1 dimensions in the large $N$ limit \cite{Sezgin:2003pt, Klebanov:2002ja}. This connection between bosonic and fermionic theories through holographic duality, was the main motivation for researchers to consider the possibility that there is a duality connecting these bosonic and fermionic theories directly. 

The existence of such a duality was first speculated by \textit{Minwalla et al\textcolor{red}{.}} \cite{Giombi:2011kc}. The statement of the duality was made precise by \textit{Aharony} \cite{Aharony:2015mjs}.  It has been shown that one can reach this duality through a \textcolor{red}{r}enormali\textcolor{red}{s}ation \textcolor{red}{g}roup (RG) flow from one of the supersymmetric dualities we discussed above \cite{Gur-Ari:2015pca}.\\

The first novelty that we stumbled upon in the work outlined in this thesis, was a ground state that had been overlooked since it only manifests itself in the finite chemical potential regime. It is a ground state with non-zero expectation value for the gauge fields, seemingly breaking rotational symmetry. In reality, colo\textcolor{red}{u}r-flavo\textcolor{red}{u}r locking remedies this and the rotational symmetry is preserved. In order to confirm that the rotational symmetry is preserved, we computed the quadratic spectrum of the theory, which showed no signs of rotational asymmetry. We discovered further, owing to Goldstone's theorem and the broken $U(1)_B$ global symmetry, that the long wave-length dynamics of the $SU(2$) theory behaved as a superfluid and predicts the existence of a roton excitation. What was even more surprising is that this non-trivial ground state exists in the zero coupling free scalar theory coupled to Chern-Simons. This is peculiar since we generally expect there to be a potential that drives the symmetry breaking.

Correlation functions for the scalar Chern-Simons theory at \textcolor{red}{chemical potential} $\mu=0$ have been computed exactly in the large $N$ limit \cite{Aharony:2012nh}. Since the $\mu =0$ results cannot be extrapolated to finite $\mu$, we would need to take the ground state discussed in the previous paragraph into account. Further, \textit{Jain et al\textcolor{red}{.}} \cite{Jain:2013gza} have computed the partition function for all values of temperature and chemical potential. However, this result again hinges on the assumption that the ground state of the system is trivial. We show this is not the case. A natural question that arises is, what do the correlation functions look like in the presence of this new condensate at large $N$? Answering this question was the ultimate aim that we had in commencing this work, \textit{i.e.} solving this theory in the large $N$ limit. In order to pursue this direction, we set ourselves the more conservative goal of computing the quadratic spectrum in the $SU(N)$ theory, for general $N$. The ground state in the $SU(N)$ theory has the same algebraic structure as the non-commutative Chern-Simons theory used as a model for the \textit{\textcolor{red}{f}ractional \textcolor{red}{q}uantum Hall effect} (FQHE). The approach to diagonali\textcolor{red}{s}ing the large $N$ spectrum of fluctuations that we outlined involves translating the problem to a non-commutative Chern-Simons theory and using the solutions of that model to arrive at an exact result.

    Aside from studying the ground state and the fundamental excitations of finite density Chern-Simons models, we have also explored the existence and properties of solitons (vortices) in these models. Our main results in this line are the discovery and study of an approximate numerical \textcolor{red}{\textit{Bogomol'nyi, Prasad \& Sommerfeld}} (BPS) state in the abelian theory. Additionally, we find a self\textcolor{red}{-}consistent vortex ansatz for the $SU(2)$ and $U(2)$ theories. We have left the numerical study of the resulting equations for future work.

    This thesis is structured in the following way:

    In Chapter \textcolor{red}{(\ref{ch:Background})}, we cover the necessary background required to understand the rest of the thesis and how it fits in the wider field of research. We present a condensed introduction to the topics of Chern-Simons field theory \textcolor{red}{\cite{Dijkgraaf:1989pz, Deser:1981wh, Dunne:1998qy, Witten:1988hf}}, the \textcolor{red}{q}uantum Hall effect \textcolor{red}{\cite{Tong:2016kpv, Hall1879, vonKlitzing:1980pdk, Landau1930, yoshioka2002the, Girvin, PhysRevLett.48.1559, Laughlin:1983fy}}, vortices \textcolor{red}{\cite{Tong:2005un,PhysRevLett.15.240,shifman2012,Abrikosov1957,Nielsen:1973cs,Paul:1986ix,Hong:1990yh,Jackiw:1990aw,Jackiw:1990pr,Kosterlitz_1973, Berezinsky:1970fr, Berezinsky:1972rfj,Bogomolny:1975de,Prasad:1975kr}}, particle-vortex duality \textcolor{red}{\cite{Peskin:1977kp,Dasgupta:1981zz,Intriligator:1996ex,Murugan:2015boa,Karch:2016sxi,Kapustin:1999ha,Burgess:2000kj,Murugan:2014sfa}}, Fermi-Bose duality \textcolor{red}{\cite{Aharony:2015mjs,Giveon:2008zn, Benini:2011mf, Aharony:2013dha, Aharony:2014uya,Aharony:2012ns,Giombi:2011kc,Jain:2013gza, Gur-Ari:2015pca}}, the statistical grand canonical ensemble in quantum field theory and non-commutative field theory \textcolor{red}{\cite{Polychronakos:2007df, Szabo:2001kg, Douglas:2001ba,Jackiw:2002pn,Snyder:1946qz, Snyder:1947nq,Connes:1994yd,Witten:1985cc,Sen:1986bh,Seiberg:1999vs,Susskind:2001fb,Polychronakos:2001mi}}.

    Chapter \textcolor{red}{(\ref{ch:Chapter_2})} presents the analytical and numerical study of abelian vortex solutions interpolating between a symmetric and asymmetric phase, where the symmetry breaking is chemical potential driven.

    Chapter \textcolor{red}{(\ref{ch:Chapter_3})} showcases the derivation of the peculiar ground state\textcolor{red}{, which we} mentioned earlier\textcolor{red}{,} and the spectrum of fluctuations is analy\textcolor{red}{s}ed in the $SU(2)$ theory. \textcolor{red}{This is followed by} the $SU(N)$ ground state and the observation that it resembles non-commutative Chern-Simons theory. \textcolor{red}{At the end of the chapter, we switch gears back to vortices, except this time we consider the possibility of topological solutions in the non-abelian theory. We show that there is a self-consistent circularly symmetric ansatz for a global vortex in the $SU(2)$ theory and a local vortex in the $U(2)$ theory.}

    



%A bunch of references that I will need (Particle-Vortex Duality) \cite{Peskin:1977kp, Dasgupta:1981zz, PhysRevB.39.2756} 
%(The first example of supersymmetric $\mathcal{N}=2$ Particle-Vortex like duality) \cite{Aharony:1997bx}
%$\mathcal{N}=2$ 3d CS matter gauge theories and flow from Giveon Kutasov to Aharony duality.\cite{Intriligator:2013lca}
%Aharony Duality \cite{Aharony:1997gp}
%Seiberg Duality in 3dimensions \cite{Karch:1997ux}
%Mirror Symmetry in $\mathcal{N}=4$ SUSY \cite{Intriligator:1996ex}
%SU(N) $\mathcal{N}=2$ large N exact free energy calculation \cite{Jain:2012qi}
%General supersymmetric 3d Chern-Simons duality \cite{Aharony:2014uya}
%Monopole operators in SUSY CS \cite{Aharony:2015pla, Borokhov:2002cg}
%\cite{Benini:2011mf, deBoer:1997kr}}
%Non-SUSY dualities have been motivated primarily by the vector model/high spin $AdS_4$ correspondence. \cite{Sezgin:2003pt, Klebanov:2002ja}, Original high spin paper \cite{Vasiliev:1992av}
%M-theory  - \cite{Witten:1995ex}
%S-duality in string theory \cite{Sen:1994fa} \cite{Schwarz:1994xn}
%T-duality in string theory  \cite{Sathiapalan:1986zb}
%U-duality in string theory \cite{Hull:1994ys}
%Original Chern-Simons form paper  \cite{Chern:1974ft}
%Topologically massive gauge theory. CS+YM \cite{Deser:1982vy, Deser:1981wh}
%Checks of Fermi-Bose Duality \cite{Aharony:2012nh, Giombi:2011kc, Aharony:2011jz}
%Statement of Fermi-Bose Duality \cite{Aharony:2015mjs}

%This will be the introduction chapter. I will talk about:
%
%
%
%\begin{itemize}
%    \item The reunion of QFT and CMT that has been sparked in recent years by studies such as dualities
%    \item Importance of topological field theory in practice. Quantum Computing and \textit{etc}. 
%    \item Most importantly, highlight the importance of studying dualities as a tool to understand strongly coupled field theories. Deforming well known dualities and attempting to match the two sides takes us closer to having dual descriptions of theories that are of physical interest.
%    \item 
%\end{itemize}
